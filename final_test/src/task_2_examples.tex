\documentclass{article}

% subfigures
\usepackage{subcaption}

% densely packed page
\usepackage{fullpage}

% force layout
\usepackage{float}

% Layout control for itemize and enumerate
\usepackage{enumitem}

% no indentation for the start of the paragraph
\setlength{\parindent}{0cm}

\begin{document}

% style config
\raggedbottom

{\Huge Fuzzy logic}

\section*{Definitions}
Each task contains sets of 1. $observations$, 2. $conclusions$, and 3. $rules$.
\\[0.5em]
The sets of $observations$ might not be defined for the same domain (though, you will notice that they often are), so they are marked with different variables:

$\mu_A(x)$ $\longleftarrow$ fuzzy set $A$ containing values of $x$ from its domain.\\
$\mu_B(y)$ $\longleftarrow$ fuzzy set $B$ containing values of $y$ from its domain.
\\[0.5em]
A set of observations will look like this:

$\mu_A(x) = \{\frac{0.0}{1}, \frac{0.1}{2}, \frac{0.4}{3}, \frac{0.7}{4}, \frac{1.0}{5}\}$
\\[0.5em]
Please note that the set does not contain fractions. This is notation describing each element's degree of belonging.
The objects of the set are read as follows: $\frac{0.4}{3}$ - "element $3$ belongs to set $A$ to a degree of $0.4$".
\\[0.5em]
As opposed to set theory you are used to, elements can "partially" belong to a set. Elements with a degree of $0.0$ fully do not belong to the set, while element of a degree of $1.0$ fully belong to the set.
The degree is always a number between $0$ and $1$, both inclusive.
\\[0.5em]
Sets of conclusions are always defined on the same domain, as they describe the same variable.
\\[0.5em]
The set of rules is defined in a similar fashion to classical logic:
\begin{enumerate} [noitemsep, nolistsep]
    \item if $x$ is $\neg A$ then $z$ is $\neg N$
    \item if $x$ is $\neg A$ $\lor$ $y$ is $B$ then $z$ is $M$
    \item if $x$ is $A$ $\land$ $y$ is $\neg B$ then $z$ is $K$
\end{enumerate}
Where variables $x$ and $y$ are inputs found in the observation sets $A$ and $B$ respectively, and $z$ is an output we are trying to determine, found in conclusion sets $N$, $M$, and $K$.
\\[0.5em]
The goal of each task is to calculate the value of $z$ for some known values of $x$ and $y$.
\\[0.5em]
Remember that your answer on the value of $z$ must be in the domain of $z$. If the value is fractional, round it to the nearest value in the domain (halfway point is rounded up).

\section*{Practice tasks}
Using the Zadeh s and t-norm and Mandami's implication, calculate the value of $z$ for given values $x$ and $y$. 

\subsection*{1}
$\mu_A(x) = \{\frac{0.0}{1}, \frac{0.1}{2}, \frac{0.4}{3}, \frac{0.7}{4}, \frac{1.0}{5}\}$ \\ [0.3em]
$\mu_B(y) = \{\frac{0.1}{0}, \frac{0.5}{25}, \frac{0.8}{50}, \frac{0.7}{75}, \frac{0}{100}\}$
\\[0.7em]
$\mu_N(z) = \{\frac{0.5}{1}, \frac{0.7}{2}, \frac{1.0}{3}, \frac{0.0}{4}, \frac{0.5}{5}\}$ \\ [0.3em]
$\mu_M(z) = \{\frac{0.2}{1}, \frac{0.3}{2}, \frac{0.9}{3}, \frac{0.7}{4}, \frac{0.0}{5}\}$ \\ [0.3em]
$\mu_K(z) = \{\frac{1.0}{1}, \frac{0.3}{2}, \frac{0.1}{3}, \frac{0.2}{4}, \frac{1.0}{5}\}$ \\

1. if $x$ is $\neg A$ then $z$ is $\neg N$ \\
2. if $x$ is $\neg A$ $\lor$ $y$ is $B$ then $z$ is $M$ \\
3. if $x$ is $A$ $\land$ $y$ is $\neg B$ then $z$ is $K$ \\

$x=4; y=0; z=?$


\subsection*{2}

$\mu_A(x) = \{\frac{0.1}{1}, \frac{0.5}{2}, \frac{0.0}{3}, \frac{0.3}{4}, \frac{0.7}{5}\}$ \\ [0.3em]
$\mu_B(y) = \{\frac{1.0}{15}, \frac{0.0}{30}, \frac{0.2}{45}, \frac{0.3}{60}\}$
\\[0.7em]
$\mu_N(z) = \{\frac{0.2}{25}, \frac{0.3}{50}, \frac{1.0}{75}\}$ \\ [0.3em]
$\mu_M(z) = \{\frac{0.0}{25}, \frac{0.5}{50}, \frac{0.7}{75}\}$ \\ [0.3em]
$\mu_K(z) = \{\frac{0.2}{25}, \frac{0.6}{50}, \frac{0.9}{75}\}$ \\

1. if $x$ is $\neg A$ $\lor$ $y$ is $B$ then $z$ is $N$ \\
2. if $x$ is $\neg A$ then $z$ is $M$ \\
3. if $x$ is $\neg A$ $\land$ $y$ is $\neg B$ then $z$ is $\neg K$ \\

$x=1; y=45; z=?$


\subsection*{3}


\end{document}